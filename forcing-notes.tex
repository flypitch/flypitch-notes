\documentclass[11pt]{article}
 
\usepackage[margin=1in]{geometry} 
\usepackage{tikz-cd}
\usepackage{graphicx}
\usepackage{hyperref}
\usepackage{amsmath,amsthm,amssymb,mathrsfs,mathabx}
\usepackage{enumitem}
\usepackage{csquotes}
\usepackage{color} 
\usepackage{parskip}
\usepackage{hyperref}
\usepackage{fixltx2e}
\usepackage{tcolorbox}
\usepackage{jmhmacros}

\graphicspath{ {/home/pv/Pictures/latex/} }

\begin{document}

\title{Forcing and the independence of the continuum hypothesis}
\author{Jesse Han}
\date{\today}

\maketitle


\begin{abstract}
In these notes, intended as the plaintext part of the Flypitch project, we give a complete account of the independence of the continuum hypothesis from $\msf{ZFC}$, with special attention paid to comparing the different approaches: generic sets, Boolean-valued models, and double-negation sheaves.
\end{abstract}

% TODO(jesse) \section*{Introduction}

\section{Preliminaries}

\subsection{First-order logic}


\subsection{The completeness theorem}

\section{Generic sets: Cohen's original proof}


\section{Boolean-valued models}


\section{Sheaves and filterquotients}


\end{document}